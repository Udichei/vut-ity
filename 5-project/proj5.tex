\documentclass[10pt, hyperref={unicode}, xcolor=dvipsnames]{beamer}


\usepackage[czech]{babel}
\usepackage[utf8]{inputenc}
\usepackage{times} 
\usepackage{listings} 
\usepackage{xcolor} 
\usepackage{graphics} 
\usepackage{booktabs}


\usetheme{Szeged}
\usecolortheme{spruce}


\definecolor{keywordpurple}{RGB}{167,29,93}
\definecolor{commentgray}{RGB}{150,152,150}
\definecolor{stringblue}{RGB}{24,54,145}


\lstset{
	basicstyle=\footnotesize,
	commentstyle=\color{commentgray},
	frame=single,
	keepspaces=true,
	keywordstyle=\color{keywordpurple},
	language=C,
	morekeywords={new, use},
	numbers=left,
	stringstyle=\color{stringblue},
	showstringspaces=false,
	showspaces=false,
	showtabs=false,
	tabsize=4,
}



\title{Typografie a~publikování\,--\,5.~projekt}
\subtitle{Binární strom}
\author{Elizaveta Syanova}
\date{\today}
\institute
{
	Vysoké učení technické v~Brně\\
	Fakulta informačních technologií
}


\defbeamertemplate*{footline}{shadow theme}
{%
  \leavevmode%
  \hbox{\begin{beamercolorbox}[wd=.5\paperwidth,ht=2.5ex,dp=1.125ex,leftskip=.3cm plus1fil,rightskip=.3cm]{author in head/foot}%
    \usebeamerfont{author in head/foot}\insertframenumber\,/\,\inserttotalframenumber\hfill\insertshortauthor
  \end{beamercolorbox}%
  \begin{beamercolorbox}[wd=.5\paperwidth,ht=2.5ex,dp=1.125ex,leftskip=.3cm,rightskip=.3cm plus1fil]{title in head/foot}%
    \usebeamerfont{title in head/foot}\insertshorttitle%
  \end{beamercolorbox}}%
  \vskip0pt%
}

\begin{document}


\maketitle

\section{Introduction}

\begin{frame}
\frametitle{Introduction}
    % číslovaný obsah
	\setbeamertemplate{section in toc}[sections numbered]
	% skrytí podsekcí v obsahu
	\tableofcontents[hideallsubsections]

\end{frame}

\section{Definice}


\begin{frame}

    \begin{columns}
    \frametitle{Definice}
    
    \column{0.5\textwidth}
        \textbf{Binární strom} je hierarchická datová struktura, ve které každý uzel má nejvýše dva potomky. 
    \column{0.5\textwidth}
    \end{columns} 
\end{frame}


\begin{frame}
    \frametitle{Typy binárních stromů}
    \setbeamercovered{transparent}
    
    \begin{itemize}
        \item<1-> \textbf{Plný binární strom} všechny jeho listy jsou ve stejné hloubce.
        \item<2-> \textbf{Úplný binární strom } každý vnitřní uzel má dva syny.
        \item<3-> \textbf{Vyvážený binární strom } hloubka listů se od sebe liší maximálně o jedna.
    \end{itemize}
\end{frame}



\begin{frame}
    Stromy se dělí na:
    
    \setbeamercovered{transparent}
    
    \begin{itemize}
        \item<1-> Uspořádané (ordered tree)
        \item<2-> Neuspořádané (unordered tree)
    \end{itemize}
\end{frame}



\section{Práce s binárním stromem}


\begin{frame}
    V praktickém programování je obvykle binární strom reprezentován dvěma způsoby: 
    
    \begin{itemize}
        \item<1-> Pomocí \alert{dynamické struktury}, kde jsou hrany reprezentovány ukazateli.
        \item<2-> Pomocí \alert{pole}.
    \end{itemize}
    
\end{frame}

\begin{frame}[fragile]
    
    \begin{lstlisting}[title={Příklad struktury:}]
    struct  TreeNode
    {
        double data; 
        TreeNode *left;  
        TreeNode *right;
    }
    \end{lstlisting}
\end{frame}




\begin{frame}[fragile]
    
    \begin{lstlisting}[title={Příklad algoritmu s poli (Řazení haldou):}]
    void razeni_haldou (double array[], int n) {
        int z; 
        int i;
        double temp;
    
        for (z = 1; z < n; z++) {
            temp = array[z];
            i = z;
        
            while (i > 0 && temp > array[PARENT(i)]) {
                array[i] = array[PARENT(i)];
                i = PARENT(i);
            }
            array[i] = temp;
        
        ...
    }
    \end{lstlisting}
\end{frame}



\section{Použité zdroje}

\begin{frame}
\frametitle{Použité zdroje}
\begin{thebibliography}{10}
		\bibitem[Stromy]{stromy} Stromy
		\newblock \texttt{http://uzlabina2.aspone.cz/stromy.aspx}

		\bibitem[Habr]{stromy} Habr: Binárné stromy
		\newblock \texttt{https://habr.com/ru/post/267855/}
		
		\bibitem[Wiki]{stromy} Wikipedia: Heapsort
		\newblock \texttt{https://en.wikipedia.org/wiki/Heapsort}
		
		\bibitem[ADT]{adt} Abstraktní datové typy
		\newblock \texttt{https://slidetodoc.com/dal-abstraktn-datov-typy-abstraktn-datov-typy-adt/}
	\end{thebibliography}
\end{frame}



\end{document}
