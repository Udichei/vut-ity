\documentclass[a4paper, twocolumn, 10pt]{article}
\usepackage[utf8]{inputenc}
\usepackage[usenames]{color}
\usepackage{indentfirst}
\usepackage[a4paper,left=1.8cm,text={18cm, 25cm},top=1.8cm]{geometry}
\usepackage[IL2]{fontenc}
\usepackage{hyperref}
\usepackage[czech]{babel}
\usepackage{amsmath}
\usepackage{dsfont}
\usepackage{amsthm}

\begin{document}
\begin{titlepage}
\begin{center}
    \huge
    \textsc{FAKULTA INFORMAČNÍCH TECHNOLOGIÍ
VYSOKÉ UČENÍ TECHNICKÉ V BRNĚ}
    \\[84mm]
    Typografie a publikování – 2. projekt

Sazba dokumentů a matematických výrazů
    \vfill
\end{center}

{\LARGE 2021 \hfill Štefan Martiček (imarticek)}

\end{titlepage}

\textbf{\LargeÚvod}

~\

V této úloze si vyzkoušíme sazbu titulní strany, matematických vzorců, prostředí a dalších textových struktur obvyklých pro technicky zaměřené texty (například rovnice (1)
nebo Definice 1 na straně 1). Rovněž si vyzkoušíme používání odkazů  \textbackslash \texttt{ref} a \textbackslash \texttt{pageref}.

Na titulní straně je využito sázení nadpisu podle optického středu s využitím zlatého řezu. Tento postup byl
probírán na přednášce. Dále je použito odřádkování se
zadanou relativní velikostí 0.4 em a 0.3 em.

V případě, že budete potřebovat vyjádřit matematickou
konstrukci nebo symbol a nebude se Vám dařit jej nalézt
v samotném \LaTeX u, doporučuji prostudovat možnosti balíku maker \AmS-\LaTeX.

\section{Matematický text}
Nejprve se podíváme na sázení matematických symbolů
a výrazů v plynulém textu včetně sazby definic a vět s využitím balíku amsthm. Rovněž použijeme poznámku pod
čarou s použitím příkazu \textbackslash \texttt{footnote}. Někdy je vhodné
použít konstrukci \textbackslash \texttt{mbox\{\}\ }, která říká, že text nemá být
zalomen

\newtheorem{theorem1}{Definice}
\begin{theorem1}
Rozšířený zásobníkový automat \textit{(RZA) je definován jako sedmice tvaru $A = (Q, \Sigma, \Gamma, \delta,  q_0, Z_0, F)$,
kde:}
\end{theorem1}

\begin{itemize}

\item \textit{Q je konečná množina} vnitřních (řídicích) stavů,

\item $\Sigma$ \textit{je konečná} vstupní abeceda

\item $\Gamma$ \textit{je konečná} zásobníková abeceda

\item $\delta$ je přechodová funkce $Q\times(\Sigma\cup\{\epsilon\})\times\Gamma^{\ast}\rightarrow 2^{Q\times\Gamma^{\ast}}$,

\item $q_0 \in Q$ je počáteční stav,$Z_0 \in \Gamma$ je startovací symbol
zásobníku a $F \subseteq Q$ \textit{je množina} koncových stavů.

\end{itemize}

Nechť $P = (Q, \Sigma, \Gamma, \delta, q_0, Z_0, F)$ je rozšířený zásobníkový automat. \textit{Konfigurací} nazveme trojici $(q, \omega, \alpha) \in Q \times \Sigma^{\ast}\times\Gamma^{\ast}$, kde $q$ je aktuální stav vnitřního řízení, $\omega$ je dosud nezpracovaná část vstupního řetězce a $\alpha = Z_{i1}Z_{i2}\ldots Z_{ik}$ je obsah zásobníku\footnote{$Z_{i1}$ je vrchol zásobníku}.

\subsection{Podsekce obsahující větu a odkaz}

\textbf{Definice 2.} Řetězec $\omega$ nad abecedou $\Sigma$ je přijat RZA \textit{A jestliže $(q_0, \omega, Z_0)\overset{\ast}{\underset{A}{\vdash}}(q_F ,\epsilon, \gamma)$ pro nějaké $\gamma \in \Gamma^{\ast}$ a $q_F \in F$. Množinu $L(A) = \{ \epsilon\mid \epsilon$ je přijat RZA A\} $\subseteq\Sigma^{\ast}$ nazýváme} jazyk přijímaný RZA $A$.

\newpage
\newtheorem{theorem2}{Věta}
Nyní si vyzkoušíme sazbu vět a důkazů opět s použitím
balíku \texttt{amsthm}.

\begin{theorem2}Třída jazyků, které jsou přijímány ZA, odpovídá \mbox{bezkontextovým jazykům.} 
\end{theorem2}

\begin{proof}{
V důkaze vyjdeme z Definice 1 a 2.
}\end{proof}
\section{Rovnice a odkazy}

Složitější matematické formulace sázíme mimo plynulý
text. Lze umístit několik výrazů na jeden řádek, ale pak je
třeba tyto vhodně oddělit, například příkazem \textbackslash \texttt{quad}.

~\

$\sqrt[i]{x_i^3}$ kde $x_i$ je $i$-té sudé číslo splňující \quad $x_i^{x_i^{i^2} + 2} \leq y_i^{x_i^4}$

~\

V rovnici (1) jsou využity tři typy závorek s různou explicitně definovanou velikostí

\begin{equation}
x = [ \{ [ a + b]\ast c\}^d \oplus 2]^{3/2}
\end{equation}

\begin{equation}\nonumber
y = \lim\limits_{x \rightarrow \infty}\frac{\frac{1}{\log_{10} x}}{\sin^{2}x + \cos^{2}x}
\end{equation}

V této větě vidíme, jak vypadá implicitní vysázení limity $\lim_{n \rightarrow \infty} f(n)$ v normálním odstavci textu. Podobně je to i s dalšími symboly jako $\prod _{i=0}^n 2^i$ či $\bigcap_{A\in B}A$ V případě vzorců $ \lim\limits_{n \rightarrow \infty} f(n)$ a $\prod\limits _{i=0}^n 2^i$
jsme si vynutili méně
úspornou sazbu příkazem \textbackslash \texttt{limits}.

\begin{equation}
\int_b^a g(x)dx = -\int\limits_a^b f(x)dx
\end{equation}

\section{Matice}
Pro sázení matic se velmi často používá prostředí \texttt{array} a závorky (\textbackslash \texttt{left}, \textbackslash \texttt{right}).

\begin{equation}\nonumber
\left(\begin{array}{ccc}
    a - b &  \widehat{\xi + \omega} & \pi \\
    \vec{\mathbf{a}} & AC & \hat{\beta}
\end{array}\right)
= 1 \Longleftrightarrow\mathcal{Q} = \mathds{R}
\end{equation}

\begin{equation}\nonumber
\mbox{\textbf{A}} =
\left\| \begin{array}{cccc}
    a_{11} &  a_{12} & \dots & a_{1n}\\
    a_{21} &  a_{22} & \dots & a_{2n}\\ 
    \vdots &  \vdots & \ddots & \vdots\\
    a_{m1} &  a_{m2} & \dots & a_{mn}\\
\end{array}\right\|
=
\left| \begin{array}{cccc}
    t &  u \\
    v &  w 
\end{array}\right|
=
tw - uw
\end{equation}

Prostředí \texttt{array} lze úspěšně využít i jinde.

\begin{equation}\nonumber
\left(\begin{array}{c}
    n \\
    k 
\end{array}\right)
= \left\{
\begin{array}{c l}
0 & \text{pro } k < 0 \text{ nebo } k > n\\
\frac{n!}{k!(n - k)!} & \text{pro } 0\leq k\leq n.
\end{array} \right.
\end{equation}


\end{document}
